\chapter{Забруднення}

\section{Основні викиди}

До основних викидів, що забруднюють атмосферу, належать:
\begin{itemize}
    \item Тверді частинки (PM)
    \item Оксиди азоту
    \item Сірчастий газ
    \item Озон
    \item Вуглекислий газ
    \item Чадний газ
\end{itemize}

\section{Тверді частинки (PM)}

PM (particulate matter) це особлива категорія викидів в атмосферу, що включає в себе всі не газоподібні забруднювачі 
з малим розміром частинок. Частинки мають різноманітний хімічний склад і деякі можуть бути 
токсичними. 

Ці викиди класифікують за розміром частинок. Найпоширенішими групами є PM10 та PM2.5, 
що відповідають частинкам з діаметром менше 10 та 2.5 мікронів відповідно. 
Подібні частинки можуть долати великі відстані в атмосфері за допомогою вітру. 
Тож деяка частина цих викидів прилітає з-за кордону. \cite{noauthor_dusts_nodate}

Невелика частина забруднення має природній характер (пил, морські частинки, вулканічний попіл) і більшість забруднення мають антропогенні джерела. 
Для PM це може бути викиди від спалення палива в транспорті, промисловості або пожеж в лісах. 
В великих містах значна частина PM з'являється від зношених шин та дисків автотранспорту. 

Концентрація PM в повітрі моніториться організаціями з охорони здоров'я і граничні 
допустимі норми регулюються законодавством, задля цього був створений індекс якості повітря 
відносно PM.\cite{noauthor__nodate}


\begin{center}
    \begin{tabular}{c c c}
        Індекс якості повітря & 
        $\mathrm{PM}_{2.5}$ & $\mathrm{PM}_{10}$ \\
        \cellcolor{green}
        Добрий & 0 & 0\\
        \cellcolor{yellow}
        Задовільний & 12 & 54\\
        \cellcolor{orange}
        Шкідливий для групи ризику & 35 & 154\\
        \cellcolor{red}
        Шкідливий & 55 & 254\\
        \cellcolor{Mahogany}
        Дуже шкідливий & 150 & 354\\
        \cellcolor{Sepia}
        Небезпечний & 250 & 424\\
    \end{tabular}
    
    \vspace{1cm}
    \labelformat{1}{}{Таблиця 1.1: Індекс якості повітря відносно PM у 
    $\frac{\text{мкг}}{\text{м}^3}$}
\end{center}


Підвищена PM зазвичай спостерігається біля доріг з інтенсивним рухом транспорту та
біля зони підприємств у великих містах. Як вже зазначалося, ці частинки можуть 
переноситися вітром тож великі концентріції заюруднення спостерігаються лише в дні з 
слабким вітром або у місцевості де вітер не виносить забруднення за межі міста.

Частинки через дихання потраплють у кровообіх і можуть осідати у внутрішніх органах, 
таким чином частинки менші 10 мікронів можуть спричинити проблеми з диханням та 
впливати на роботу серця.\cite{noauthor_dusts_nodate}

\section{Оксиди азоту (NOx)}

Більша частина оксидів азоту утворюється в результаті з'єднання кисню з азотом у полум'ї. 
Менша частина є результатом горіння сполук азоту в паливі. Природньо NOx утворюється в наслідок 
блискавки і незначною мірою мікробних процесів в ґрунті.

Антропогенні викиди оксидів домінують серед інших викидів за массою. Лише в британції ці викиди 
становлять близько 2.2 мільйонів тон кожyого року. З них половина припадає на транспорт, 
чверть на електростанції і решта на інші промислові та побутові процеси спалювання.\cite{noauthor_nitrogen_nodate}

\begin{center}
    \begin{tikzpicture}
        \pie[hide number, text = legend]
        {
            33.9/Road traffic,
            17.9/ther mobile sources,
            14.1/Industry and waste,
            13.4/Inland navigation,
            10.1/Energy sector,
            3.6/Households,
            2.6/Agriculture,
            2.3/Other static sources,
            2.1/Aviation
            }
    \end{tikzpicture}
\end{center}
\labelformat{2}{}{Рис 1.1 Діаграма чинників викидів оксидів азоту за 2022 рік по світу}

\vspace{0.75cm}



Основними забруднювачами данного типу є оксид азоту $\mathrm{NO}$ та меншою мірою 
діоксид азоту $\mathrm{NO}_2$. Протягом дня ці оксиди в атмосфері перетворюються один в одного. 
Оксид азоту окислюється в атмосфері до $\mathrm{NO}_2$ за участю озону протягом десятків 
хвилин, а діоксид розщеплюється під діє ультрафіолетового випромінювання на $\mathrm{NO}$ та 
атом оксигену, що утворює озон з киснем. Таким чином ці гази існують у квазірівноважному стані 
за участі світла. Згодом діоксид азоту окислюється до азотної кислоти, яка швидко поглинається 
під час контакту з поверхнями, назважаючи на це самі оксиди зникають повільно і 
можуть подорожувати на великі відстані до розкладу на кислоту або нітрати. Тож забруднення 
однієї країни спричиняють забруднення і в сусідніх. Найбільша концентрація 
спостігається в районах великих міст через які проходять автомагістралі з інтенсивним рухом.\cite{noauthor_nitrogen_nodate}

Високий рівень діоксиду азоту може спричинити пошкодження дихальних шляхів та 
підвищити вразливість людини до респіраторних інфекцій і астми.
Тривалий вплив може спричинити хронічні захворювання легенів.


\section{Сірчастий газ (SO2)}

Діоксид сірки виділяється при спалюванні палива, що містить сірку, тож основними джерелами 
данних викидів є виробництво електроенергії, промислове та побутове спалення палива. Міжнародні 
організації зменшують кількість викидів сірчистого газу за допомогою законів і врегулювань. 
Також було розроблене обладнення для очищення димових газів від сірки і завдяки йому викиди 
продовжують зменшуватись, не звачаючи на збільшення використання вугілля з 2000-них років.
Агенством з охорони навколишнього середовища був створений індекс якості повітря відносно 
сірчистого газу, що дозволяють визначити наскільки небезпечна концентрація викидів для 
здоров'я людини:


\begin{center}
    \begin{tabular}{c c}
        Індекс якості повітря & 
        Частка в повітрі (ppm) \\
        \cellcolor{green}
        Добрий &  0 \\
        \cellcolor{yellow}
        Задовільний & 0,1\\
        \cellcolor{orange}
        Шкідливий для групи ризику & 0,2\\
        \cellcolor{red}
        Шкідливий & 1,0 \\
        \cellcolor{Mahogany}
        Дуже шкідливий & 3.0\\
        \cellcolor{Sepia}
        Небезпечний & 5.0 \\
    \end{tabular}
    
    \vspace{1cm}
    \labelformat{1}{}{Таблиця 1.2: Індекс якості повітря відносно SO2 у ppm (мільйонна частка)}
\end{center}



Короткостроковий контакт з викидами може спричинити проблеми з дихальною 
системою людини. Особливо небезпечними подібні контакти для групи ризику - 
люди з астмою або діти. Також викиди діоксиду спричиняють формуванню інших 
оксидів сірки, що в наслідок реакціїї з іншими компоннтами атмосфери можуть 
перетворитись на тверді частки (PM), що в свою чергу вже мають свої 
наслідки для здоров'я людини.\cite{noauthor_sulphur_nodate}


Коли сірковий газ реагує з повітрям і водою, утворюється корозійна рідина - 
сіркова кислота, це одна з головних компонент кислотних дощів, що 
спричиняють велику шкоду навколишньому середовищу. На додачу сам сірковий 
газ сповільнює ріст рослин і пошкоджує листя. 


При взаємодії сіркового газу з карбоном утворюються сульфатні айрозолі, що 
збільшують життя хмар сприяючі глобальному потеплінню.


\section{Озон(O3)}

Озон корисний високо, поганий поблизу. Озоновий шар, який знаходиться високо у верхніх шарах атмосфери, захищає нас від значної частини ультрафіолетового випромінювання Сонця. Проте забруднене озоном повітря на рівні землі, де ми можемо ним дихати, спричиняє серйозні проблеми зі здоров’ям. Озон агресивно атакує легеневу тканину, вступаючи з нею в хімічну реакцію. \cite{noauthor__nodate-1}

Приземний озон утворюється в атмосфері з газів, які викадються із вихлопних труб, димових труб, 
фабрик та багатьох інших джерел забруднення. Коли ці гази контактують із сонячним світлом, 
вони реагують і утворюють озоновий дим. Для утворення озону необхідні оксиди азоту, легкі органічні сполуки та сонячне світло. Утворення оксидів азоту було розглянено в одному з 
попередніх підрозділів. Легкі органічні сполуки викидаються в повітря з деяких звичайних 
споживчих товарів, таких як фарба, і коли випаровуються побутові хімікати, такі як розріджувачі 
фарб і розчинники. вони також викидаються від автомобілів, хімічних заводів, нафтопереробних 
заводів, фабрик і автозаправних станцій. За присутності цих газів в правильних умовах 
утворююється озон і згодом вітер може рознести утворені викиди на великі відстані. 
Високий рівень озону частіше спостерігається влітку через високі температури і оскільки 
підвищення рівня данного виду викидів сприяє потеплінню це утворює циклічну залежність. \cite{___2010,association_ozone_nodate}

У Галвестоні, Техас, було проведене дослідження, яке показало, що навіть короткочасний вплив 
озону може погіршити здоров'я дрослих людей. Дослідження показало, що рятувальники мали більшу 
обструкцію легенів наприкінці дня, коли рівень озону був високий.

Групи ризику для впливу озону:

\begin{itemize}
    \item вагітні жінки;
    \item діти;
    \item люди старші за 65 років;
    \item люди з астмою або іншими захворюваннями дихальних шляхів;
    \item Люди з нижчим соціально-економічним статусом;
    \item люди, які працюють або займаються спортом надворі.
\end{itemize}

Деякі контраверсійні дослідження показують більший вплив саме на жінок, але на данний момент 
відсутній остаточній консенсус стосовно цього питання.

Вплив озону у поєднанні з іншими факторами ризику скорочує середню тривалість життя. Існують 
переконливі докази смертоносності довгострокового впливу озону завдяки маштабним дослідженням. 
Було встановлено, що ризик передчасної смерті зростає зі збільшенням рівня озону. \cite{us_epa_health_2015,admin_ozone_2022}

В багатьох країнах в літній період утворюється достатньо озону для того щоб викликати проблеми 
зі здоров'ям. Інші проблеми крім скорочення середньої тривалості життя включають в себе: 

\begin{itemize}
    \item задишка хрипи і кашель
    \item респіраторні інфекції
    \item сприятливість до запалення легенів
    \item потреба в госпіталізації групи ризику
\end{itemize}

По мірі збільшення тривалості впливу озону можуть також з'являтися і інші проблеми. Це може бути 
метаболічні розлади, проблеми нервової системи, репродуктивні проблеми, рак, а також збільшення 
смертності від сердцевих захвонювань. 

Вдихання інших викидів може зробити аргонізм більш вразливи до озону і навпаки - вдихання озону 
підвищить реакцію на інші забруднювачі. Вплив озону також може підсилити реакцію у людей з 
алергією.\cite{us_epa_ground-level_2015}

Нові дослідження натякають на те, що можливо варто переосмислити стандарти щодо оцінки данного
забруднення. Наприклад, дослідження 2017 року продемонструвало, що люди похолого віку мають 
симптоми навіть коли рівень озону залишається нижчим за поточний національний стандарт.


\section{Вуглекислий газ (CO2)}

Вуглекислий газ це важливий для землі газ, що зберігає тепло в атмосфері, що вивільняється за спалювання викопного палива. 
Також велика кількість вуглекислого газу викидається в атмосферу в результаті природніх процесів таких, як дихання та виверження вулканів, тож позбутися нанівець від цього газу неможливо. 
Хоча вуглекислий газ не вважається забрудником, через те що він є натуральним компонентом атмосфери та всі живі організми викидають його, але його кількість в атмосфері сильно зросла через спалювання викопаного палив у виробницві. 
Невеликі концентрації його у повітрі безпечні для дихання, проте вони створюють парниковий ефект і сприяють глобальному потеплінню.

Найпоширеніший антропогенний чинник данного виду викидів полягає у спалюванні викопаного палива - вугілля, нафти та газу, для отримання тепла, електроенергії, руху транспорту.
Крім прикладу електростанцій, що перетворюють тепло на електроенергію, можна навести будівничі процеси розповсюджені з приходом урбанізації - під час виробництва цементу використовується велика кількість викопаного палива для спалювання матеріалів цементу. 
Під час цього спалювання відбуваються хімічні реакції, що вивільняють вуглекислий газ.
Загалом будівничі процеси багаті на хімічні реакції, що вивільняють вуглекислий газ, також вони залучають використання великої кількості транспорту, що також викидає вуглекислий газ.
Крім того, вирубка лісів також вивільняє накопичений вуглець з лісових ландшафтів в атмосферу. 
Обернений до цього процес - секвестрація поліпшує стан справ і є одним із найперспективніших рішень для данного виду викидів. 

Загалом вплив цих чинників виглядає так:



\begin{center}
    \begin{tikzpicture}
        \pie[text = legend]
        {
            32/Промисловість,
            28/Будівничі процеси,
            23/Транспорт,
            11/Будівничі матеріали і будівництво,
            6/Інші
        }
    \end{tikzpicture}
\end{center}
\labelformat{2}{}{Рис 1.2 Діаграма антропогенних чинників викидів вуглекислого газу за 2019 рік по світу}

\vspace{0.75cm}

Як було зазначено вище, невелика концентрація вуглекислого газу безпечна для дихання, проте великі концентрації можуть спричинити такі проблеми зі здоров'ям як: 


\begin{multicols}{2}
    \begin{itemize}
        \item Головні болі
        \item Запаморочення
        \item Неспокійність
        \item Відчуття «поколювання».
        \item Утруднене дихання
        \item Пітливість
        \item Втома
        \item Почастішання пульсу
        \item Підвищений артеріальний тиск
        \item Кома
        \item Асфіксія
        \item Судоми
    \end{itemize}
\end{multicols}
    
\section{Чадний газ (CO)}


Чадний газ, також відомий як оксид вуглецю, утворюється в результаті неповного згоряння палив, що містить у своєму складі вуглець, наприклад бензину, природного газу або деревини. 
Тож його джерелами є автомобілі, електростанції, лісові пожежі та застарілі сміттєспалювальні заводи.
За кількістю викидів переважають саме автомобілі, що викидають близько 60\% всіх викидів чадного газу.\cite{us_epa_basic_2016}
Оксид вуглецю також може утворюватися в результаті фотохімічних реакцій в атмосфері з метанових і неметанових вуглеводнів, інших летких органічних вуглеводнів в атмосфері та органічних молекул у поверхневих водах і ґрунтах. \cite{noauthor_carbon_nodate}

Ранні ознаки легкого та помірного отруєння CO схожі на грип або харчове отруєння (за винятком відсутності температури), і деякі загальні симптоми включають:

\begin{multicols}{2}
    \begin{itemize}
        \item Головний біль
        \item Запаморочення 
        \item Нудота
        \item Задишка 
        \item Втома 
    \end{itemize}
\end{multicols}

Більш високі рівні отруєння призводять до гірших симптомів, включаючи нудоту, втруту свідомості, кому та навіть іноді призводять до летальних випадків.\cite{meersens_carbon_2022}
