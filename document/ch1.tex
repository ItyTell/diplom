\chapter{Забруднення}

\section{Основні викиди}

До основних викидів, що забруднюють атмосферу, належать:
\begin{itemize}
    \item Тверді частинки (PM)
    \item Оксиди азоту
    \item Сірчастий газ
    \item Озон
    \item Вуглекислий газ
    \item Чадний газ
\end{itemize}

\section{Тверді частинки (PM)}

PM (particulate matter) це особлива категорія викидів в атмосферу, що включає в себе всі не газоподібні забруднювачі 
з малим розміром частинок. Частинки мають різноманітний хімічний склад і деякі можуть бути 
токсичними. 

Ці викиди класифікують за розміром частинок. Найпоширенішими групами є PM10 та PM2.5, 
що відповідають частинкам з діаметром менше 10 та 2.5 мікронів відповідно. 
Подібні частинки можуть долати великі відстані в атмосфері за допомогою вітру. 
Тож деяка частина цих викидів прилітає зза кордону. 

Невелика частина забруднення має природній характер (пил, морські частинки, вулканічний попіл) і більшість забруднення мають антропогенні джерела. 
Для PM це може бути викиди від спалення палива в транспорті, промисловості або пожеж в лісах. 
В великих містах значна частина PM з'являється від зношених шин та дисків автотранспорту.

Концентрація PM в повітрі моніториться організаціями з охорони здоров'я і граничні 
допустимі норми регулюються законодавством, задля цього був створений індекс якості повітря 
відносно PM.

\begin{center}
    \begin{tabular}{c c c}
        Індекс якості повітря & 
        $\mathrm{PM}_{2.5}$ & $\mathrm{PM}_{10}$ \\
        \cellcolor{green}
        Добрий & 0 & 0\\
        \cellcolor{yellow}
        Задовільний & 12 & 54\\
        \cellcolor{orange}
        Шкідливий для групи ризику & 35 & 154\\
        \cellcolor{red}
        Шкідливий & 55 & 254\\
        \cellcolor{Mahogany}
        Дуже шкідливий & 150 & 354\\
        \cellcolor{Sepia}
        Небезпечний & 250 & 424\\
    \end{tabular}
    
    \vspace{1cm}
    \labelformat{1}{}{Таблиця 1.1: Індекс якості повітря відносно PM у 
    $\frac{\text{мкг}}{\text{м}^3}$}
\end{center}


Підвищена PM зазвичай спостерігається біля доріг з інтенсивним рухом транспорту та
біля зони зони підприємств у великих містах. Як вже зазначалося, ці частинки можуть 
переноситися вітром тож великі концентріції заюруднення спостерігаються лише в дні з 
слабким вітром або у місцевості де вітер не виносить забруднення за межі міста.

Частинки через дихання потраплють у кровообіх і можуть осідати у внутрішніх органах, 
таким чином частинки менші 10 мікронів можуть спричинити проблеми з диханням та 
впливати на роботу серця.

\section{Оксиди азоту (NOx)}

Більша частина оксидів азоту утворюється в результаті з'єднання кисню з азотом у полум'ї. 
Менша частина є результатом горіння сполук азоту в паливі. Природньо NOx утворюється в наслідок 
блискавки і незначною мірою мікробних процесів в ґрунті.

Антропогенні викиди оксидів домінують серед інших викидів за массою. Лише в британції ці викиди 
становлять близько 2.2 мільйонів тон кожнаого року. З них половина припадає на транспорт, 
чверть на електростанції і решта на інші промислові та побутові процеси спалювання.

Основними забруднювачами данного типу є оксид азоту $\mathrm{NO}$ та меншою мірою 
діоксид азоту $\mathrm{NO}_2$. Протягом дня ці оксиди в атмосфері перетворюються один в одного. 
Оксид азоту окислюється в атмосфері до $\mathrm{NO}_2$ за участю озону протягом десятків 
хвилин, а діоксид розщеплюється під діє ультрафіолетового випромінювання на $\mathrm{NO}$ та 
атом оксигену, що утворює озон з киснем. Таким чином ці гази існують у квазірівноважному стані 
за участі світла. Згодом діоксид азоту окислюється до азотної кислоти, яка швидко поглинається 
під час контакту з поверхнями, назважаючи на це самі оксиди зникають повільно і 
можуть подорожувати на великі відстані до розкладу на кислоту або нітрати. Тож забруднення 
однієї країни спричиняють забруднення і в сусідніх. Найбільша концентрація 
спостігається в районах великих міст через які проходять автомагістралі з інтенсивним рухом.

Високий рівень діоксиду азоту може спричинити пошкодження дихальних шляхів та 
підвищити вразливість людини до респіраторних інфекцій і астми.
Тривалий вплив може спричинити хронічні захворювання легенів.



\section{Чинники забруднення, види і класифікація}



\section{Методи вимірювання забруднення}

\section{Методи протидії}
