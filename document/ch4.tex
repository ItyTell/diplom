\chapter{Висновки}

У даній роботі було проведено комплексний аналіз для прогнозування рівня викидів різних забруднювачів у атмосферу. 
Спочатку було зібрано та оброблено дані, що включали різні екологічні та метеорологічні параметри. 
Було проведено аналіз даних, що включав в себе візуалізацію та виявлення залежностей між різними параметрами і перетворення даних для покращення результатів прогнозування.

Для інтерполяції було застосовано дві моделі: лінійна регресія та нейронна мережа.

Модель лінійної регресії показала високу точність на даних CO2 та озону, але продемонструвала певне зниження точності відносно PM10 та PM2.5. 

Нейронна мережа продемонструвала високу точність для CO2, озону. 
Проте для PM10 та деяких оксидів модель виявилася неефективною, що вказує на слабку залежність прогнозованих даних від обраних параметрів. 
Похибки для цих викидів на тренувальних данних були малими. 
Це вказує на те що модель підлаштувалась під шум в даних і не виявила реальні залежності.

Результати порівняння з наївним алгоритмом показали, що для деяких забруднювачів наша модель перевершує простий підхід, тоді як для інших виявлено необхідність вдосконалення моделі та виявлення додаткових параметрів, що дійсно впливають на рівень цих викидів.

Загалом, проведений аналіз показав, що використання як лінійної регресії, так і нейронних мереж має великий потенціал для прогнозування рівня викидів забруднювачів у атмосферу, для яких мають розподіл подібний до нормального, адже саме озон, що підпадає під цей критерій показав найкращі результати у прогнозуванні в обох моделях. 
Однак для досягнення високої точності необхідно продовжувати дослідження, вдосконалювати моделі та враховувати більше релевантних факторів.