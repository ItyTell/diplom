\chapter{Висновки}

У даній роботі було проведено комплексний аналіз для прогнозування рівня викидів різних забруднювачів у атмосферу. 
Спочатку були зібрані та оброблені дані, що включали різні екологічні та метеорологічні параметри. 
Були проведені аналіз даних, що включав в себе візуалізацію та виявлення залежностей між різними параметрами і перетворення даних для покращення результатів прогнозування.

Для інтерполяції були застосовані дві моделі: лінійна регресія та нейронна мережа.

Модель лінійної регресії показала високу точність на даних CO2 та озону, але продемонструвала певне зниження точності відносно PM10 та PM2.5. 

Нейронна мережа продемонструвала високу точність для CO2, озону. 
Проте для PM10 та деяких оксидів модель виявилася не ефективною, що вказує на слабку залежність прогнозованих даних від обраних параметрів. 
Похибки для цих викидів на тренувальних данних були малими. 
Це вказує на те що модель мідлаштувалась під шум в даних і не виявила реальні залежності.

Результати порівняння з наївним алгоритмом показали, що для деяких забруднювачів наша модель перевершує простий підхід, тоді як для інших виявило необхідність вдосконалення моделі та виявлення додаткових параметрів, що дійсно впливають на рівень цих викидів.

Загалом, проведений аналіз показав, що використання як лінійної регресії, так і нейронних мереж має великий потенціал для прогнозування рівня викидів забруднювачів у атмосферу, даня яких мають розподіл подібний до нормального, адже саме озон, що підпадає під цей критерій показав найкращі результати в прогнозування в обох моделях. 
Однак для досягнення високої точності необхідно продовжувати дослідження, вдосконалювати моделі та враховувати більше релевантних факторів.