\chapter*{ВСТУП}

Забруднення повітря є дуже важливим питанням для здоров'я людей та навколишнього середовища.
Воно може бути викликане різними факторами, такими як викиди транспорту, промисловрості та 
сільського господарства. На стан повітря певної місцевості також може впливати вітер, 
температура та географія самого міста. Велика кількість забруднюючих речовин може викликати 
різні захворювання та проблеми зі здоров'ям, тому це є важливим фактором для вибору місця 
проживання і відповідно важливою проблемою для влади міста, яка піклується про доброжиток 
населення міста.


Моніторинг і можливість прогнозування забруднення повітря є необхідним у сучасному світі. 
Для забезпечення здоров'я та комфорту населення міста важливо мати точні дані, щодо забруднення 
та його динаміки. Також важливо мати можливість прогнозувати забруднення задля вивчення 
аномалій та можливості вчасно реагувати на різкі зміни у стані повітря. Відмінності між 
реальним станом повітря і прогнозованим може бути наслідком появи нового чинника забрудення 
і за допомогою аналізу подібних відмінностей ці чинники можуть бути вчасно зупинені.
Крім того моделі можуть продемонструвати, як на стан повітря вплинули покращувальні заходи,
що дозволить вибрати найбільш ефективні методи боротьби з забрудненням. 
