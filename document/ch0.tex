\chapter*{ВСТУП}

{\bf{Актуальність теми.}}
Забруднення повітря є дуже важливим питанням для здоров'я людей та навколишнього середовища.
Воно може бути викликане різними факторами, такими як викиди транспорту, промисловрості та 
сільського господарства. На стан повітря певної місцевості також може впливати вітер, 
температура та географія самого міста. Велика кількість забруднюючих речовин може викликати 
різні захворювання та проблеми зі здоров'ям, тому це є важливим фактором для вибору місця 
проживання і відповідно важливою проблемою для влади міста, яка піклується про доброжиток 
населення міста.


Моніторинг і можливість прогнозування забруднення повітря є необхідним у сучасному світі. 
Для забезпечення здоров'я та комфорту населення міста важливо мати точні дані, щодо забруднення 
та його динаміки. Також важливо мати можливість прогнозувати забруднення задля вивчення 
аномалій та можливості вчасно реагувати на різкі зміни у стані повітря. Відмінності між 
реальним станом повітря і прогнозованим може бути наслідком появи нового чинника забрудення 
або впливу зміни клімату і за допомогою аналізу подібних відмінностей ці чинники можуть 
бути визначені.
Крім того моделі можуть продемонструвати, як на стан повітря вплинули покращувальні заходи,
що дозволить з меншими похибками визначати, які з методів є кращими. 

{\bf{Мета і завдання роботи.}}  
Метою даної дипломної роботи є розробка та застосування моделей для прогнозування розповсюдження шкідливих викидів у атмосфері, зокрема за допомогою лінійної регресії та нейронних мереж. Це дозволить покращити точність прогнозування рівнів забруднення повітря та забезпечити своєчасне реагування на можливі аномалії в забрудненні.

Завданнями роботи є:
\begin{itemize}
    \item Збір та обробка данних: зібрати дані стосовно забруднення та параметрів за допомогою, яких буде виконуватись прогнозування. Провести очистку та підготовку даних для аналізу.
    \item Аналіз даних: Вивчити розподіл даних, провести логарифмічне перетворення та нормалізацію, за потреби, для підвищення точності моделей. Виявити патерни та взаємозв'язки між параметрами.
    \item Побудова моделей: Розробити та навчити моделі лінійної регресії та нейронної мережі для прогнозування викидів забруднювачів у повітря. Оптимізувати моделі для досягнення найкращих результатів.
    \item Прогнозування: Здійснити прогнозування рівнів забруднення та порівняти точність різних моделей.
    \item Оцінка та валідація: Оцінити точність моделей.
\end{itemize}

{\bf{Об'єкт дослідження.}} 
Об'єктом дослідження данної кваліфікаційної роботи є процеси розповсюдження шкідливих викидів у атмосфері та їх взаємодія з кліматичними факторами, такими як вітер, температура та географія місцевості. 


{\bf{Об'єкт дослідження.}} 
У данній роботі були використані такі методи:
\begin{itemize}
    \item Статистичний аналіз: Для вивчення розподілу даних та виявлення основних патернів і аномалій.
    \item Візуалізація даних: Створення графіків і гістограм для наочного представлення розподілу даних та результатів моделювання.
    \item Моделювання: Розробка та застосування моделей лінійної регресії та нейронних мереж для прогнозування рівнів забруднення повітря.
\end{itemize}

