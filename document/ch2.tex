\chapter{Вирішення задачі}

\section{Дані}

Дані відносно викидів були скачані з офіційного сайту європейського союзу за період 2019-2023 року.
Кліматичні дані за той же період були взяті з сайту НАСА. 
Формат данних - NETCDF, що зберігає дані в багатовимірних масивах з прив'язкою до географічних координат.


\section{Інструменти}

Для обробки даних було використано мову програмування Python на веб-інтерактивній обчислювальній платформі Jupyter Notebook. 
Для анлізу та візуальзації данних були застосовані такі бібліотеки як: 

\begin{itemize}
    \item numpy - для роботи з масивами даних
    \item NETCDF4 - для роботи з NETCDF файлами
    \item matplotlib - для візуалізації даних
    \item tensorflow - для побудови нейронної мережі
\end{itemize}


